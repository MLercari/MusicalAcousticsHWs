\documentclass[a4paper]{article}
\usepackage[utf8]{inputenc}
\usepackage[english]{babel}
\usepackage{amsmath} % per ambienti tipo cases
\usepackage{amssymb}
\usepackage{mathtools}
\usepackage{siunitx}
\usepackage{graphicx} % per includere figure
%\usepackage{subfigure}
\usepackage{booktabs} % per le tabelle
\usepackage{caption}
\usepackage{fancyhdr}
\usepackage{hyperref}
\usepackage[section]{placeins}
\usepackage{microtype}
\usepackage{caption}
\usepackage{subcaption}
\captionsetup[subfigure]{labelfont=rm}
\usepackage{verbatim} %multiline comments
\usepackage[backend=biber, style=numeric, safeinputenc, sorting=none]{biblatex}
\addbibresource{source.bib}	% uncomment for bibliography



%opening
\title{}
\author{}

\pagestyle{fancy}
\lhead{Musical Acoustics}
\chead{HL3}
\rhead{10743504, 10751919}
\newcommand{\Rarrow}{\mbox{\Large$\Rightarrow$}}

\begin{document}

\begin{titlepage}	
	\newcommand{\HRule}{\rule{\linewidth}{0.5mm}} % Defines a new command for horizontal lines, change thickness here
	
	\center % Centre everything on the page
	
	%------------------------------------------------
	%	Headings
	%------------------------------------------------
	
	\includegraphics[width=.4\textwidth]{Logo_Politecnico_Milano.png}\\[0.4cm]
	\textsc{\LARGE}\\[0.3cm] % Main heading such as the name of your university/college
	
	\textsc{\large MSc. Music and Acoustic Engineering}\\[1cm] % Minor heading such as course title
	
	\textsc{\Large Musical Acoustics - A.Y. 2020/2021}\\[0.5cm] % Major heading such as course name
	
	%------------------------------------------------
	%	Title
	%------------------------------------------------
	
	\HRule\\[0.4cm]
	
	{\huge\bfseries HL3 – Modeling Techniques}\\[0.4cm] % Title of your document
	
	\HRule\\[1.5cm]
	
	
	
	{\large\textit{Authors' IDs:}}\\
	10743504, 10751919, % Your name
	%\\ \textsc{Gruppo 11}
	
	%------------------------------------------------
	%	Date
	%------------------------------------------------
	
	\vfill\vfill\vfill % Position the date 3/4 down the remaining page
	
	{\large\today} % Date, change the \today to a set date if you want to be precise
	
	%------------------------------------------------
	%	Logo
	%------------------------------------------------
	
	\vfill\vfill
	%\includegraphics[width=0.2\textwidth]{Politecnico_di_Milano.eps}\\[1cm] % Include a department/university logo - this will require the graphicx package
	
	%----------------------------------------------------------------------------------------
	
	\vfill % Push the date up 1/4 of the remaining page
	
	
\end{titlepage}
\section{Comments on the FD implementation}
\subsection{Simulation data}

The code contains an implementation of a backward finite difference scheme which is used to simulate a piano string struck with a felt hammer. The string under consideration is tuned to C2 ($f_0 = \SI{52.8221}{\hertz}$). We chose a sampling frequency of $f_s = \SI{176.4}{\kilo\hertz}$ as suggested in ... and it's well above the Nyquist frequency so we don't expect aliasing in the time domain. Other dimensions that are not suggested in the assignment, such as string mass, hammer mass, string length etc. have been taken from ... and can be found listed in Tab. \ref{tab:vals} . The string has a linear mass $\mu = M_s / L$ and the tension applied on it at rest has been computed as $T_0 = 4L^2f_0^2\mu$ which determines that the speed of sound along the string is $c = \sqrt{T_0/\mu}$. The spatial sampling has been chosen under the limit imposed by the stability condition ... $$ N_{max} = \sqrt{\frac{-1 + \sqrt{1 + 16 \epsilon \gamma ^2}}{8 \epsilon}} $$ where $\gamma = f_s / 2 f_0 $. The number of spatial steps has been set as $N = N_{max} -1 = 537$ and the relative spatial resolution is $X = \SI{0.0036}{\meter}$. 

 \begin{table}[h]
	\centering
	$\begin{array}{llc}
		\toprule
		\text{string length} &	L & \SI{1.92}{\meter}\\
		\text{string mass}   &	M_s & \SI{35e-3}{\kilogram}\\
		\text{string stiffness parameter} &	\epsilon & 7.5 	\times 10^{-6}\\
		\text{string stiffness coefficient} &	\kappa & 7.5 	\times 10^{-6}\\
		\text{relative striking position} &	a & 0.12 \\
		\midrule
		\text{viscous damping coefficient} & b_h & \SI{1e-4}{\second^{-1}}\\
		\text{hammer mass} & M_h & \SI{4.9e-3}{\kilogram} \\
		\text{hammer stiffness} &K & \SI{4e8}{\newton\per\meter}\\
		\text{stiffness exponent} &	p & 2.3 \\
		\midrule
		\text{hinge normalized impedance} &	\zeta_l & \SI{e20}{\second^3\meter^2\per\kilogram^2} \\
		\text{bridge normalized impedance} &	\zeta_b & \SI{e3}{\second^3\meter^2\per\kilogram^2} \\		
		\bottomrule
	\end{array}$
	\caption{Values considered in the simulation.}
	\label{tab:vals}
\end{table}

\printbibliography


\end{document}